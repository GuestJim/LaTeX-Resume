\usepackage{xcolor}

\newcounter{box}
%	need a counter for if it is odd or even line
\newcounter{boxC}
%	to separate the factor for coloring the boxes and the actual counter

\def\colorodd{white!0}
\def\coloreven{lightgray}
%	sets default colors for the odd and even lines in the zebra boxes

\newcommand{\zcolor}[2][white!0]{
	\def\colorodd{#1}
	\def\coloreven{#2}
}
%	allows the line colors to be changed
%		changing the odd line color is optional

\usepackage{calc}
\newcounter{colorstep}
\setcounter{colorstep}{1}
\newcommand{\zstep}[1]{
	\setcounter{colorstep}{#1}
}
%	for controlling the stepping between stripes, using the calc package

\newcommand{\zbox}[2][0.975]{
	\setcounter{boxC}{\value{box} / \value{colorstep} + 1}
%		using the calc package, this will divide the box counter by the step and because it only returns the integer part, odd and even checks work
%
	\stepcounter{box}
%		need to increment the counter second, so it starts at 0 for the coloring
%	
	\ifodd	\value{boxC}
		{\colorbox{\colorodd}{\parbox{#1\linewidth}{#2}}}%
	\else
		{\colorbox{\coloreven}{\parbox{#1\linewidth}{#2}}}%
	\fi
%	the \ifodd command is from original TeX, which is why \value{series} does not have brackets surrounding it
%	the % is needed at the end of the line to prevent extra vertical space from being added
}
%	this is the actual command to draw the zebra boxes and by using the counters, it does this automatically

\newcommand{\zitem}[2][0.975]{
	\zbox[#1]{\item	#2}
}
%	will put the zbox in a list

\newcommand{\zbullet}[2][0.95]{
	\stepcounter{box}
	
	\ifodd	\value{box}
		\parbox{#1\linewidth}{\textbullet	\hspace{0.25em}	{\colorbox{\colorodd}{\parbox{#1\linewidth}{#2}}}}%
	\else
		\parbox{#1\linewidth}{\textbullet	\hspace{0.25em}	{\colorbox{\coloreven}{\parbox{#1\linewidth}{#2}}}}%
	\fi
}
%	for when an itemized list will not work with zbox

\newcommand{\zclear}{
	\setcounter{box}{0}
}
%	clears the counter for zbox, so the coloring will reset between objects